% !TeX spellcheck = en_US
%
%
\only<article>{
	%
	\maketitle
	\thispagestyle{empty}
	\setcounter{page}{0}
	%
	\newpage
	\tableofcontents
}

%%
\begin{frame}[c]
\textbf{\large A Short Note on Programming} ~\\~\\% 
\begin{center}
	~\\~\\
	...and specifically on Python.~\\~\\~\\
	\colorbox{white}{\includegraphics[width=0.5\textwidth]{media/Python_logo.png}}
\end{center}
\end{frame}


%%
\begin{frame}[c]
\textbf{\large Programming Languages in Numerical Mathematics (selection)}\\~\\
\begin{itemize}
	\item
	\textbf{Fortran} (FORmula TRANslation) (1957):
	
	\begin{itemize}
		
		\item
		proprietary (e.g. from IBM) and free compilers
		\item
		intended for numerical calculations (matrix and vector operations)
		\item
		extensive libraries
		\item
		LAPACK (\textbf{L}inear \textbf{A}lgebra \textbf{Pack}age) standard
		library for numerical linear algebra
	\end{itemize}
\end{itemize}
\begin{itemize}
	\item
	\textbf{C} (1972)/ \textbf{C++} (1985):
	\begin{itemize}
		
		\item
		universal programming language
		\item
		Standard libraries for numerics: Armadillo, LAPACK++ (based on
		LAPACK)
	\end{itemize}
\end{itemize}
%
\begin{itemize}
	
	\item
	\textbf{MATLAB} (MATrix LABoratory) (1984):
	
	\begin{itemize}
		
		\item
		proprietary software from MathWorks
		\item
		designed for numerical mathematics (matrix and vector operations)
	\end{itemize}
\end{itemize}
%
\begin{itemize}
	\item
	\textbf{Mathematica} (1988):
	\begin{itemize}
		
		\item
		proprietary software from Wolfram Research
		\item
		visualization of 2d/3d objects
		\item
		symbolic processing of equations
		\item
		see also: \url{https://www.wolframalpha.com/}
	\end{itemize}
\end{itemize}

\begin{itemize}
	
	\item
	\textbf{Python} (1990): open source
	
	\begin{itemize}
		
		\item
		universal programming language (several application areas)
		\item
		for numerical calculations: SciPy (2001), NumPy (1995,2006),
		matplotlib (2003).
	\end{itemize}
\end{itemize}

\begin{itemize}
	
	\item
	\textbf{Julia} (2012): open source
	
	\begin{itemize}
		
		\item
		developed mainly for scientific computing
		\item
		syntax looks like MATLAB
		\item
		execution speed is in the range of C and Fortran
	\end{itemize}
\end{itemize}
\end{frame}

\begin{frame}[c]
$\rightarrow$ All programming related parts of this lecture will be presented and implemented using \textbf{Python 3}
\Vspace{1cm}
\textbf{Why Python?}
\Vspace{0.5cm}
\begin{itemize}	
	\item
	universal, multi-purpose programming language	
	\begin{itemize}		
		\item
		many packages for scientific computing, web development, ...
	\end{itemize}
	\item
	open source and free (Python Software Foundation License (PSFL), OSI/FSF approved)
	\item
	multi-platform (runs on all operating systems)
	\item
	design philosophy: easy syntax, readable code (almost looks like
	pseudocode)
	\item Also see: \url{https://www.youtube.com/watch?v=M0vBoBqqjr0}
\end{itemize}
\end{frame}

%
%
\begin{frame}[c]
\textbf{\large Background}
\Vspace{0.5cm}
\begin{minipage}{0.7\textwidth}
	
\begin{itemize}	
	\item
	developed in 1990 by Guido van Rossum (Netherlands)	
	\begin{itemize}	
		\item
		name is homage to Monty Python
	\end{itemize}
	~\\\item
	interpreter (scripting) programming language\\ ($\neq$
	compiled language such as C or Fortran)
	~\\[0.7cm]\item
	used by: Google Mail, Google Maps, YouTube, Dropbox, sphinx, and many more
	~\\[0.7cm]\item
	for scientific computing we use from the Scipy Stack:
	\begin{itemize}
		\item \textbf{NumPy} (1995,2006)
		\item \textbf{SciPy} (2001)
		\item \textbf{matplotlib} (2003)
	\end{itemize} 
\end{itemize}
\end{minipage}
%\begin{minipage}{1\textwidth}
%	\includegraphics[width=0.3\textwidth]{media/life-of-brian.jpg}
%\end{minipage}
\end{frame}


%
%
\begin{frame}[c]
\textbf{\large Programming Workflow}
~\\~\\
\textbf{CLI} \demo{demo}

\begin{itemize}
	
	\item Any text editor can be used (emacs, vi, vim, nano, geany,
	gedit,...)
	\begin{itemize}
		\item many editors provide syntax highlighting
	\end{itemize}
	\item Install Python and then interpret the source code
\end{itemize}
~\\~\\
	\textbf{IDE} \demo{demo}
\begin{itemize}
	\item For software development it is often more convenient to use an
\textbf{integrated development environment (IDE)}
	\item Specifically for Python: PyCharm\footnote{sign up to jetbrains with your university
	account and you can get the PyCharm professional edition!}, Spyder
\end{itemize}
\end{frame}

%
%
\begin{frame}[c]
	For exercise submission/presentation: 
\Vspace{0.5cm}
 \textbf{Jupyter-Notebook} \demo{demo}
\hspace*{0.5cm}
\begin{itemize}
	\item
	open source, \emph{web based} interactive environment\\
	$\rightarrow$ thus multi-platform
	\item
	developed by   \textbf{Project Jupyter} (NPO)
	~\\
	$\rightarrow$ name refers to: \textbf{Ju}lia, \textbf{Pyt}hon, \textbf{R}
\end{itemize}

\begin{itemize}
	\item
	the whole process can be documented:\\
	Coding $\rightarrow$ Documentation $\rightarrow$ Run
	$\rightarrow$ Communication and Presentation
\end{itemize}

\begin{itemize}
	\item
	in fact, a jupyter notebook contains all the input \textbf{and} output
	of an interactive session plus additional text
	$\rightarrow$ complete record!
\end{itemize}

\begin{itemize}
	\item client VS server
	\begin{itemize}
		\item client (local lightweight machine): browser-based workflow
		\item server (remote, number cruncher): does the actual computation
	\end{itemize}
\end{itemize}
\end{frame}

\begin{frame}[c]
\textbf{\large Get
			Started}
~\\~\\~\\
\begin{itemize}
	\item
	We recommend to download the distribution \textbf{\emph{Anaconda}}:
	\begin{center}
		\url{https://www.anaconda.com/distribution/}
	\end{center}
\Vspace{0.5cm}
	
	$\rightarrow$ available for Linux, Windows, and MacOS
\end{itemize}
\Vspace{0.5cm}

\begin{itemize}
	
	\item
	Comes along with:
	
	\begin{itemize}
		
		\item
		graphical user interface (\emph{Anaconda Navigator})
		\item
		Spyder, Jupyter Notebook, RStudio (IDE for R)
		\item
		installs all important packages (NumPy, SciPy, matplotlib,
		TensorFlow, scikit-learn,\$\textbackslash{}ldots\$)
		\item
		package manager (\emph{Conda}) (standard is \emph{pip})
	\end{itemize}
\end{itemize}
\end{frame}


\begin{frame}[c]
\textbf{Tutorials}
~\\~\\
\begin{itemize}
	\item Scientific computing with Python:\\
	 \url{https://scipy-lectures.org/}~\\~\\
	\item Official Online-Documentation:\\
	\url{https://docs.python.org/3/}~\\~\\
	\item Official Python Tutorial:\\
	\url{https://docs.python.org/3/tutorial/index.html}~\\~\\
	\item
	Quickstart to Jupyter Notebook:
	\\{\centering\url{https://jupyter.readthedocs.io/en/latest/content-quickstart.html}}~\\~\\
\end{itemize}
%
~\\
\textbf{Final remark}:
\begin{itemize}
	\item
	For Software development I would always go with an IDE due to the many
	additional tools: debugging, variable explorer, version control, file manager, etc.
	\item
	However, Jupyter Notebooks are very well suited for presentations and
	thus teaching. In particular for mandatory submissions, since the
	tutor can see your output, even if the program does not run on his/her
	machine (for whatever reasons).
\end{itemize}
\end{frame}
